\documentclass[a4paper]{article}
%\documentclass[a4paper,twocolumn]{article}
%\documentclass[a4paper,landscape,twocolumn]{article}
%\documentclass[a4paper]{memoir}
\usepackage[utf8]{inputenc}
\usepackage[spanish, es-tabla, es-nodecimaldot]{babel}
%\usepackage[english]{babel}
\usepackage{amsmath}  %permite usar \text{} en el entorno Matemática
\usepackage{amssymb} % para el de números reales
%\usepackage{fancyhdr} %para encabezados y pies de página lindos
%\usepackage{lastpage} %para poder referenciar el número de la última página
\usepackage{graphicx} %para insertar gráficos
\usepackage{float} %para que funcione el H de la posición de las figuras
%\usepackage{chngpage} %para cambiar márgenes temporalmente. Por ejemplo tabla o figura un poco más grande que el text width
%\usepackage[format=plain, indention=0cm, font=small, labelfont=bf, labelsep=period, textfont=sl]{caption} %tuneado del caption de las figuras
\usepackage{mathtools} % para usar dcases, la versión displayMath de cases (funciones partidas)
\usepackage{enumerate} %para personalizar los enumeradores
\usepackage{framed} %para poner párrafos adentro de un caja con marco
\usepackage{fullpage}
%\usepackage[cm]{fullpage}
\usepackage{array}
\usepackage{hyperref}
\usepackage{epigraph}
\usepackage{wrapfig} %para poner tablas o figuras con texto alrededor.
\usepackage{bbm} %para la función indicadora
\usepackage{enumitem} %para poner títulos y seguir la numeración
\usepackage{multicol}
\usepackage{enumitem}
\usepackage[font=small,labelfont=bf, labelsep=period, width=.45\textwidth]{caption} %tuneado de captions en figuras
\usepackage[nottoc]{tocbibind} %incluir referencias en el toc (table of contents, i.e. índice)



\newcolumntype{x}[1]{>{\centering\arraybackslash\hspace{0pt}}p{#1}}

%\setlength{\columnseprule}{0.5pt}
\setlength{\columnsep}{1cm}

\setlength{\epigraphwidth}{.38\textwidth}
%\setlength{\epigraphwidth}{0.7\textwidth}

\DeclareMathOperator*{\argmax}{argmax} % thin space, limits underneath in displays


%\usepackage{amssymb,amsmath,amsthm,latexsym,epsfig,euscript,multicol}
%\usepackage{enumitem}
%\usepackage[utf8x]{inputenc}

% Caracteres especiales
%\def\A{\mathbb{A}}
\def\C{\mathbb{C}}
\def \N{\mathbb{N}}
%\def \P{\mathbb{P}}
%\def \Q{\mathbb{Q}}
\def \R{\mathbb{R}}
%\def \Z{\mathbb{Z}}
\def \sen{\textrm{sen}}

%\theoremstyle{definition}
%\newtheorem{ejer}{Ejercicio}
%\newcommand{\bej}{\begin{ejer}}
%	\newcommand{\fej}{\end{ejer}}

%\def\dt{\Delta t}
%\def\dx{\Delta x}

\def\eps{\varepsilon}
%\def\wt{$\widetilde{a}$}

\newcommand{\gp}{\ensuremath{\mathcal{GP}}}



\title{Procesos gaussianos: metodología y aplicaciones\\ Taller de Tesis I - Entrega 1\\ Maestría en Explotación de Datos y Descubrimiento del Conocimiento -- UBA 2023 cuat. 1}% \\
%\textbf{VERSIÓN PRELIMINAR}}
\author{G. Sebastián Pedersen \\ {\small sebasped@gmail.com}}

\date{Inicio: Lun 27 de Marzo de 2023. Última actualización: \today}




\renewcommand{\arraystretch}{1.3}  %para que las celdas de las tablas sean un poco más altas y entre bien el Q moño.




\begin{document}
%\pagestyle{headings}
%\markright{John Smith\hfill On page styles\hfill}

	
\maketitle

\tableofcontents	
	%\begin{abstract}
	%	Se propone un primer modelo, clásico y sencillo, que intenta capturar la idea de cuándo un estudiante entra en riesgo de deserción de una materia.
	%\end{abstract}
	%\tableofcontents



\section{Dataset, pregunta y objetivo}
\subsection{Conjunto de datos}
\label{subsec:dataset}
Los datos elegidos son de producción de petróleo y gas mensual desde 2006 publicada por la Secretaría de Energía de la Nación. Los mismos contienen el detalle mensual de la producción de petróleo, gas y agua, discriminada principalmente por pozo, yacimiento, cuenca, concesión y provincia, entre otras cosas. Los datos son abiertos y accesibles desde: 

\url{https://datos.gob.ar/dataset/energia-produccion-petroleo-gas-por-pozo-capitulo-iv}

La elección de este conjunto de datos en parte es motivada porque el autor trabajó en la industria petrolera durante algunos años, y por lo tanto posee cierto conocimiento y familiaridad con la industria y el negocio.


\subsection{Pregunta y objetivos}
\label{subsec:pregunta}
Las preguntas u objetivos son dos, una de caracter metodológico y otro de caracter práctico, y ambas hacen uso de una técnica en particular, con lo cual resulta adecuado mencionarla desde un principio: los procesos gaussianos (ver \cite{gramacy, tobar, rasmussen, murphy}). 

Por un lado hay una pregunta u objetivo metodológico sobre los procesos gaussianos que involucrará ahondar en ellos y entenderlos, darles un marco teórico adecuado e investigar sus potencialidades y limitaciones, al menos desde un punto de vista teórico. Con esto se espera tener una idea metodológica adecuada de los procesos gaussianos, tanto de su formulación teórica como de sus potenciales aplicaciones. La relevancia de este objetivo radica en que al no ser los procesos gaussianos parte de la currícula del posgrado, es una oportunidad para extrapolar conocimientos y habilidades adquiridos en otros temas y volcarlos al estudio de una técnica no antes estudiada.

Por otro lado se investigará en concreto las bondades de los procesos gaussianos para realizar predicciones sobre series temporales, tomando como ejemplo práctico en particular el dataset mencionado en \ref{subsec:dataset}. La relevancia de esta pregunta radica en que intentar predecir la producción de hidrocarburos en la Argentina es de suma importancia en tanto y en cuanto es una materia prima indispensable para cubrir las necesidades energéticas del país, y por lo tanto tener una estimación de su producción tiene consecuencias en varias área estratégicas, a saber: planeamiento energético interno, exportación como commodity, seguimiento del desarrollo de la explotación de recursos naturales, planeamiento de necesidades en infraestructura (gasoductos, maquinaria, etc.) entre otras.


\subsection{Técnicas}
Como ya se mencionó en \ref{subsec:pregunta} la técnica principal serán los procesos gaussianos. Como uno de los objetivos es estudar metodológicamente a los procesos gaussianos, y el otro es investigar la potencialidad de los mismos para realizar predicciones sobre series temporales, resulta adecuado emplear otras técnicas a modo de benchmark para poder comparar tanto resultados predictivos prácticos, asi como diferencias y similitudes entre los marcos teóricos, y poder entonces entender mejor desde los dos puntos de vista por qué la técnica es más o menos efectiva, al menos como comparación relativa contra otras. La propuesta de técnicas a emplear es:
 
\begin{itemize}
	\item Modelos naive: mean y drift, por ejemplo. (ver \cite{hyndman} sec. 5.2 )
	\item Modelos GAM (modelos aditivos generalizados): STL (Seasonal and Trend descomp usando Loess, ver \cite{hyndman} secs. 3.6 y 12.1), Prophet (ver \cite{hyndman} sec. 12.2).
	\item Procesos gaussianos (ver \cite{gramacy, tobar, rasmussen, murphy}).
\end{itemize}






\begin{thebibliography}{9}
	
	\bibitem{gramacy}
	Robert B. Gramacy, \textit{Gaussian process modeling, design and optimization for the applied sciences},\\
	\url{https://bookdown.org/rbg/surrogates/}\\
	2023-03-05.
	
	\bibitem{tobar}
	Felipe Tobar, \textit{Aprendizaje de máquinas. Cap. 8: procesos gaussianos},\\
	\url{https://raw.githubusercontent.com/GAMES-UChile/Curso-Aprendizaje-de-Maquinas/master/notas_de_clase.pdf}\\
	2021.
	
	\bibitem{rasmussen}
	Carl E. Rasmussen; Christopher K. I. Williams, \textit{Gaussian Processes for Machine Learning},\\
	\url{http://gaussianprocess.org/gpml/}\\
	MIT Press, 2006.
	
	\bibitem{murphy}
	Kevin P. Murphy, \textit{Machine Learning: A Probabilistic Perspective. Ch. 15: Gaussian processes},\\
	\url{http://noiselab.ucsd.edu/ECE228/Murphy_Machine_Learning.pdf}\\
	MIT Press, 2012.
	
	\bibitem{hyndman}
	Rob J. Hyndman; George Athanasopoulos, \textit{Forecasting: Principles and Practice},\\
	\url{https://otexts.com/fpp3/}\\
	3ra. ed., 2021.
	
\end{thebibliography}


\end{document}
